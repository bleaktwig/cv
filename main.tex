\documentclass[a4paper]{cv-style}
\sethyphenation[variant=british]{english}{}

\usepackage{hyperref}
\hypersetup{colorlinks=false}

\begin{document}

\header{Bruno}{Benkel}
\lastupdated

% === SIDEBAR ==================================================================
\begin{aside}
    \section{\href{https://bleaktwig.github.io/cv/}{Web version}}
    \section{Contact}
        Via del Grano 57/A, piano terra
        Roma 00172
        Italy
        ~
        (+39) 345 837 0792
        ~
        \href{mailto:bruno.benkel@gmail.com}{bruno.benkel
        @gmail.com}
        ~
        \href{https://github.com/bleaktwig}{github.com/bleaktwig}
    %
    \section{Languages}
        \textbf{Native or Bilingual}:
        Spanish, English
        ~
        \textbf{Professional Working}:
        Italian
        ~
        \textbf{Elementary}:
        Norwegian, Chinese
    %
    \section{Programming}
        \textbf{Expert}:
        C, Python, Java
        ~
        \textbf{Proficient}:
        Bash, C++, Haskell
        ~
        \textbf{Elementary}:
        JS, MIPS Assembly, R
    %
    \section{Tools \& Libraries}
        Linux
        Git
        Python data analysis (numpy, keras, etc.)
        CERN ROOT
        EPICS
        \LaTeX{}
        QT + PySide2
        Gimp \& Inkscape
    %
\end{aside}

% === WORK EXPERIENCE ==========================================================
\section{Employment}
    \begin{entrylist}
        \entry
            {2023--Now}
            {INFN Tor Vergata}
            {Rome, Italy}
            {\jobtitle{Postdoc EIC dRICH Online Reconstruction Developer}
            \begin{itemize}
                \item
                    Developing a neural network based online reconstruction algorithm for the dual Ring Imaging Cherenkov (dRICH) detector for the Electron Ion Collider (EIC) collaboration.
                \item
                    Assisted the EIC micro-RWELL detector group in the taking of technical decisions and developed tracking software.
                \item
                    Acted as a teacher assistant in Italian on the Big Data laboratory experiences for the Roma Tor Vergata PCTO group.
            \end{itemize}} \\
        \entry
            {2023}
            {Upwork}
            {Remote}
            {\jobtitle{Freelance Software Developer}
            \begin{itemize}
                \item
                    Developed a test framework to detect faults in small-scale bank transactions in Java.
            \end{itemize}} \\
        \entry
            {2018--2023}
            {CCTVal}
            {Valpara\'iso, Chile}
            {\jobtitle{HEP Software Developer + R\&D Computer Engineer}
            \begin{itemize}
                \item
                    Developed the standard High Energy Physics (HEP) analysis toolset used by CLAS12 Run Group E.
                \item
                    Lead the Forward Micromegas Tracker alignment team, developing both detector alignment and offline reconstruction software.
                \item
                    Lead a Drift Chambers reconstruction software optimization team, and optimized a Runge Kutta 4 implementation to improve computing speed.
                \item
                    Developed an EPICS-based slow controls system for the CLAS12 Run Group E's double target system.
                \item
                    Lead a team in the development of configuration, calibration, alarms, and front-end systems for a mechanical ventilator software in the face of a shortage during the COVID19 pandemic.
                \item
                    Developed blockchain-based software for a small-scale energy democratization project.
            \end{itemize}} \\
        \entry
            {2020--2021}
            {EntrepreneurX}
            {Remote}
            {\jobtitle{Technology Consultant}
            \begin{itemize}
                \item
                    Developed BDM: a blockchain-based payment processing technology.
                \item
                    Provided design and technology advice on projects.
            \end{itemize}} \\
        \entry
            {2020}
            {Upwork}
            {Remote}
            {\jobtitle{Freelance Software Developer}
            \begin{itemize}
                \item
                    Built a python API to connect input from an electroencephalography scanner to that of an eye movement sensor.
                \item
                    Developed software for a python application that connects several IoT devices from different manufacturers.
            \end{itemize}} \\
        \entry
            {2018}
            {UTFSM}
            {Valpara\'iso, Chile}
            {\jobtitle{Artificial Intelligence UTA}
            \begin{itemize}
                \item
                    Supervised and evaluated students' code for the course's projects.
                \item
                    Assisted the teacher in test design and evaluation.
            \end{itemize}} \\
        \entry
            {2017--2018}
            {w/ Ben Tatum}
            {Los Angeles, US}
            {\jobtitle{Field Service Engineer}
            \begin{itemize}
                \item
                    Installed electricity and computer networks at residential and commercial buildings in Los Angeles.
            \end{itemize}} \\
        \entry
            {2016--2018}
            {UTFSM}
            {Valpara\'iso, Chile}
            {\jobtitle{Computer Architecture UTA}
            \begin{itemize}
                \item
                    Lead the course's teaching assistants.
                \item
                    Taught weekly practical classes in tandem with professor's theoretical ones.
                \item
                    Designed, supervised, and evaluated monthly hardware laboratory work.
            \end{itemize}}
    \end{entrylist}

% === EDUCATION ================================================================
\section{Education}
    \begin{entrylist}
        \entry
            {2021--Now}
            {M.Sc. {\normalfont in High Energy Physics}}
            {UTFSM, Chile}
            {\vspace{-0.3cm}}
        \entry
            {2017--2019}
            {Title (M.Sc. equivalent) {\normalfont in Computer Engineering}}
            {UTFSM, Chile}
            {\vspace{-0.3cm}}
        \entry
            {2012--2017}
            {B.Eng. {\normalfont in Computer Engineering}}
            {UTFSM, Chile}
            {\vspace{-0.3cm}}
    \end{entrylist}

\pagebreak
% === PUBLICATIONS =============================================================
\section{Publications}
    \begin{entrylist}
        \entry
            {2023}
            {{\normalfont Co-author in} ~Double-pion electroproduction off protons in deutreium: quasi-free cross sections and final state interactions}
            {CLAS Collaboration}
            {\vspace{-0.3cm}}
        \entry
            {2023}
            {{\normalfont Co-author in} ~Beam spin asymmetry measurements of deeply virtual $\pi^0$ production with CLAS12}
            {CLAS Collaboration}
            {\vspace{-0.3cm}}
        \entry
            {2023}
            {{\normalfont Co-author in} ~Strong interaction physics at the luminosity frontier with 22 GeV electrons at Jefferson Lab}
            {CLAS Collaboration}
            {\vspace{-0.3cm}}
        \entry
            {2023}
            {{\normalfont Co-author in} ~Measurement of the helicity asymmetry E for the $\gamma p \rightarrow p\pi^0$ reaction in the resonance region}
            {CLAS Collaboration}
            {\vspace{-0.3cm}}
        \entry
            {2022}
            {{\normalfont Co-author in} ~First CLAS12 measurement of DVCS beam-spin asymmetries in the extended valence region}
            {CLAS Collaboration}
            {\vspace{-0.3cm}}
        \entry
            {2022}
            {{\normalfont Co-author in} ~A multidimensional study of the structure function ratio $\sigma_{LT}'\sigma_0$ from hard exclusive pi+ electro-production off protons in the GPD regime}
            {CLAS Collaboration}
            {\vspace{-0.3cm}}
        \entry
            {2022}
            {{\normalfont Co-author in} ~First Measurement of $\Lambda$ Electroproduction off Nuclei in the Current and Target Fragmentation Regions}
            {CLAS Collaboration}
            {\vspace{-0.3cm}}
        \entry
            {2022}
            {{\normalfont Co-author in} ~Observation of Correlations between Spin and Tranverse Momenta in Back-to-Back Dihadron Production at CLAS12}
            {CLAS Collaboration}
            {\vspace{-0.3cm}}
        \entry
            {2022}
            {{\normalfont Co-author in} ~Alignment of the CLAS12 central hybrid tracker with a Kalman Filter \hspace{1.5cm}}
            {CLAS Collabration}
            {\vspace{-0.3cm}}
        \entry
            {2022}
            {{\normalfont Co-author in} ~Observation of Azimuth-Dependent Suppression of Hadron Pairs in Electron Scattering off Nuclei}
            {CLAS Collaboration}
            {\vspace{-0.3cm}}
        \entry
            {2022}
            {{\normalfont Co-author in} ~Exclusive $\pi^-$ Electroproduction off the Neutron in Deuterium in the Resonance Region}
            {CLAS Collaboration}
            {\vspace{-0.3cm}}
        \entry
            {2022}
            {{\normalfont Co-author in} ~Beam-Recoil Transferred Polarization in $K^+Y$ Electroproduction in the Nucleon Resonance Region with CLAS12}
            {CLAS Collaboration}
            {\vspace{-0.3cm}}
    \end{entrylist}

% ==============================================================================
\vspace{0.4cm}
For further details, you are welcome to check the web version at \href{https://bleaktwig.github.io/cv/}{bleaktwig.github.io/cv/}.

\end{document}
